\documentclass[12pt,-letter paper]{article}
\usepackage{siunitx}
\usepackage{setspace}
\usepackage{gensymb}
\usepackage{xcolor}
\usepackage{caption}
%\usepackage{subcaption}
\doublespacing
\singlespacing
\usepackage[none]{hyphenat}
\usepackage{amssymb}
\usepackage{relsize}
\usepackage[cmex10]{amsmath}
\usepackage{mathtools}
\usepackage{amsmath}
\usepackage{commath}
\usepackage{amsthm}
\interdisplaylinepenalty=2500
%\savesymbol{iint}
\usepackage{txfonts}
%\restoresymbol{TXF}{iint}
\usepackage{wasysym}
\usepackage{amsthm}
\usepackage{mathrsfs}
\usepackage{txfonts}
\let\vec\mathbf{}
\usepackage{stfloats}
\usepackage{float}
\usepackage{cite}
\usepackage{cases}
\usepackage{subfig}
%\usepackage{xtab}
\usepackage{longtable}
\usepackage{multirow}
%\usepackage{algorithm}
\usepackage{amssymb}
%\usepackage{algpseudocode}
\usepackage{enumitem}
\usepackage{mathtools}
%\usepackage{eenrc}
%\usepackage[framemethod=tikz]{mdframed}
\usepackage{listings}
%\usepackage{listings}
\usepackage[latin1]{inputenc}
%%\usepackage{color}{   
%%\usepackage{lscape}
\usepackage{textcomp}
\usepackage{titling}
\usepackage{hyperref}
%\usepackage{fulbigskip}   
\usepackage{tikz}
\usepackage{graphicx}
\lstset{
  frame=single,
  breaklines=true
}
\let\vec\mathbf{}
\usepackage{enumitem}
\usepackage{graphicx}
\usepackage{siunitx}
\let\vec\mathbf{}
\usepackage{enumitem}
\usepackage{graphicx}
\usepackage{enumitem}
\usepackage{tfrupee}
\usepackage{amsmath}
\usepackage{amssymb}
\usepackage{mwe} % for blindtext and example-image-a in example
\usepackage{wrapfig}
\graphicspath{{figs/}}
\providecommand{\mydet}[1]{\ensuremath{\begin{vmatrix}#1\end{\vmatrix}}}
\providecommand{\myvec}[1]{\ensuremath{\begin{bmatrix}#1\end{\bmatrix}}}
\providecommand{\cbrak}[1]{\ensuremath{\left\{#1\right\}}}
\providecommand{\brak}[1]{\ensuremath{\left(#1\right)}}
\providecommand{\pr}[1]{\ensuremath{\Pr\left(#1\right)}}
\providecommand{\qfunc}[1]{\ensuremath{Q\left(#1\right)}}
\providecommand{\sbrak}[1]{\ensuremath{{}\left[#1\right]}}
\providecommand{\lsbrak}[1]{\ensuremath{{}\left[#1\right]}}
\providecommand{\rsbrak}[1]{\ensuremath{{}\left.#1\right}}
\providecommand{\brak}[1]{\ensuremath{\left(#1\right)}}
\providecommand{\lbrak}[1]{\ensuremath{\left(#1\right.}}
\providecommand{\rbrak}[1]{\ensuremath{\left.#1\right)}}
\providecommand{\cbrak}[1]{\ensuremath{\left\{#1\right\}}}
\providecommand{\lcbrak}[1]{\ensuremath{\left\{#1\right.}}
\providecommand{\rcbrak}[1]{\ensuremath{\left.#1\right\}}}
\begin{document}
\centering
\section*{Algebra}	
\begin{enumerate}


\item If the sum of zeroes of the polynomial $ p\brak x = 2x^2 - k\sqrt2x+1 $ is${\sqrt2} $,then value of k is:
\begin{enumerate}
    \item $ \sqrt{2} $
    \item $2$
    \item $ 2  \sqrt{2} $
    \item $ \frac{1}{2} $
 \end{enumerate}

\item If the roots of the equation $ax^2 + bx + c = 0$,$a \neq 0$ are real and equal, then which of the following relations is true?
\begin{enumerate}    
    \item $a = \frac{b^2}{c}$
    \item $b^2 = ac$                                                                                    \item $ac = \frac{b^2}{4}$
    \item $c = \frac{b^2}{a}$
\end{enumerate}

\item In an A.P., if the first term $a = 7$, $n$th term $a_{n} = 84$, and the sum of the first $n$ terms $s_{n} = \frac{2093}{2}$, then $n$ is equal to:
\begin{enumerate}
    \item $22$
    \item $24$
    \item $23$
    \item $26$
\end{enumerate}

\item The zeroes of a polynomial $x^2 + px + q$ are twice the zeroes of the polynomial $4x^2 - 5x - 6$. The value of $p$ is:
	\begin{enumerate}   
\item $-\frac{5}{2}$
    \item $\frac{5}{2}$
    \item $-5$
    \item $10$
	\end{enumerate}
\newpage
 \item In the given figure, graphs of two linear equations are shown. The pair of these linear equations is:
\begin{figure}[!ht]
\centering
\includegraphics[width=\columnwidth]{Image1.jpg}
\caption{}
\label{fig:enter-label}
\end{figure}
\begin{enumerate}
    \item consistent with a unique solution.
    \item consistent with infinitely many solutions.
    \item inconsistent.
    \item inconsistent but can be made consistent.
\end{enumerate}
\end{enumerate}
\end{document}
