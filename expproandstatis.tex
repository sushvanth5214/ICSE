\documentclass[12pt,-letter paper]{article}
\usepackage{siunitx}
\usepackage{setspace}
\usepackage{gensymb}
\usepackage{xcolor}
\usepackage{caption}
%\usepackage{subcaption}
\doublespacing
\singlespacing
\usepackage[none]{hyphenat}
\usepackage{amssymb}
\usepackage{relsize}
\usepackage[cmex10]{amsmath}
\usepackage{mathtools}
\usepackage{amsmath}
\usepackage{commath}
\usepackage{amsthm}
\interdisplaylinepenalty=2500
%\savesymbol{iint}
\usepackage{txfonts}
%\restoresymbol{TXF}{iint}
\usepackage{wasysym}
\usepackage{amsthm}
\usepackage{mathrsfs}
\usepackage{txfonts}
\let\vec\mathbf{}
\usepackage{stfloats}
\usepackage{float}
\usepackage{cite}
\usepackage{cases}
\usepackage{subfig}
%\usepackage{xtab}
\usepackage{longtable}
\usepackage{multirow}
%\usepackage{algorithm}
\usepackage{amssymb}
%\usepackage{algpseudocode}
\usepackage{enumitem}
\usepackage{mathtools}
%\usepackage{eenrc}
%\usepackage[framemethod=tikz]{mdframed}
\usepackage{listings}
%\usepackage{listings}
\usepackage[latin1]{inputenc}
%%\usepackage{color}{   
%%\usepackage{lscape}
\usepackage{textcomp}
\usepackage{titling}
\usepackage{hyperref}
%\usepackage{fulbigskip}   
\usepackage{tikz}
\usepackage{graphicx}
\lstset{
  frame=single,
  breaklines=true
}
\let\vec\mathbf{}
\usepackage{enumitem}
\usepackage{graphicx}
\usepackage{siunitx}
\let\vec\mathbf{}
\usepackage{enumitem}
\usepackage{graphicx}
\usepackage{enumitem}
\usepackage{tfrupee}
\usepackage{amsmath}
\usepackage{amssymb}
\usepackage{mwe} % for blindtext and example-image-a in example
\usepackage{wrapfig}
\graphicspath{{figs/}}
\providecommand{\mydet}[1]{\ensuremath{\begin{vmatrix}#1\end{\vmatrix}}}
\providecommand{\myvec}[1]{\ensuremath{\begin{bmatrix}#1\end{\bmatrix}}}
\providecommand{\cbrak}[1]{\ensuremath{\left\{#1\right\}}}
\providecommand{\brak}[1]{\ensuremath{\left(#1\right)}}
\providecommand{\pr}[1]{\ensuremath{\Pr\left(#1\right)}}
\providecommand{\qfunc}[1]{\ensuremath{Q\left(#1\right)}}
\providecommand{\sbrak}[1]{\ensuremath{{}\left[#1\right]}}
\providecommand{\lsbrak}[1]{\ensuremath{{}\left[#1\right]}}
\providecommand{\rsbrak}[1]{\ensuremath{{}\left.#1\right}}
\providecommand{\brak}[1]{\ensuremath{\left(#1\right)}}
\providecommand{\lbrak}[1]{\ensuremath{\left(#1\right.}}
\providecommand{\rbrak}[1]{\ensuremath{\left.#1\right)}}
\providecommand{\cbrak}[1]{\ensuremath{\left\{#1\right\}}}
\providecommand{\lcbrak}[1]{\ensuremath{\left\{#1\right.}}
\providecommand{\rcbrak}[1]{\ensuremath{\left.#1\right\}}}

\begin{document}
\begin{enumerate}

\item If the probability of a player winning a game is 0.79, then the probability of his losing the same game is:
\begin{enumerate}    
    \item $1.79$
    \item $0.31$
    \item $0.21	$                                                           
    \item $0.21$
\end{enumerate}

\item From the data $1, 4, 7, 9, 16, 21, 25$, if all the even numbers are removed, then the probability of getting at random a prime number from the remaining is:
	\begin{enumerate}
\item $\frac{2}{5}$
    \item $\frac{1}{5}$
    \item $\frac{1}{7}$
    \item $\frac{2}{7}$
	\end{enumerate}

\item For some data $x_{1}, x_{2}, \dots, x_{n}$ with respective frequencies $f_{1}, f_{2}, \dots, f_{n}$, the value of $\sum_{i}^{n}f_{i} \brak{x_{i} - \overline{x}}$ is equal to:
	\begin{enumerate}    
\item $n \overline{x}$
    \item $1$
    \item $\Sigma f_{i}$
    \item $0$
	\end{enumerate}

\item The middle-most observation of every data arranged in order is called:
	\begin{enumerate}    
\item mode
    \item median
    \item mean
    \item deviation
\end{enumerate}
\newpage
\item Two dice are rolled together. The probability of getting a sum of numbers on the two dice as $2$, $3$, or $5$ is:
	\begin{enumerate}    
\item $\frac{7}{36}$
    \item $\frac{11}{36}$
    \item $\frac{5}{36}$
    \item $\frac{4}{9}$
	\end{enumerate}
\end{enumerate}
\end{document}
