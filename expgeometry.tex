\documentclass[12pt,-letter paper]{article}
\usepackage{siunitx}
\usepackage{setspace}
\usepackage{gensymb}
\usepackage{xcolor}
\usepackage{caption}
%\usepackage{subcaption}
\doublespacing
\singlespacing
\usepackage[none]{hyphenat}
\usepackage{amssymb}
\usepackage{relsize}
\usepackage[cmex10]{amsmath}
\usepackage{mathtools}
\usepackage{amsmath}
\usepackage{commath}
\usepackage{amsthm}
\interdisplaylinepenalty=2500
%\savesymbol{iint}
\usepackage{txfonts}
%\restoresymbol{TXF}{iint}
\usepackage{wasysym}
\usepackage{amsthm}
\usepackage{mathrsfs}
\usepackage{txfonts}
\let\vec\mathbf{}
\usepackage{stfloats}
\usepackage{float}
\usepackage{cite}
\usepackage{cases}
\usepackage{subfig}
%\usepackage{xtab}
\usepackage{longtable}
\usepackage{multirow}
%\usepackage{algorithm}
\usepackage{amssymb}
%\usepackage{algpseudocode}
\usepackage{enumitem}
\usepackage{mathtools}
%\usepackage{eenrc}
%\usepackage[framemethod=tikz]{mdframed}
\usepackage{listings}
%\usepackage{listings}
\usepackage[latin1]{inputenc}
%%\usepackage{color}{   
%%\usepackage{lscape}
\usepackage{textcomp}
\usepackage{titling}
\usepackage{hyperref}
%\usepackage{fulbigskip}   
\usepackage{tikz}
\usepackage{graphicx}
\lstset{
  frame=single,
  breaklines=true
}
\let\vec\mathbf{}
\usepackage{enumitem}
\usepackage{graphicx}
\usepackage{siunitx}
\let\vec\mathbf{}
\usepackage{enumitem}
\usepackage{graphicx}
\usepackage{enumitem}
\usepackage{tfrupee}
\usepackage{amsmath}
\usepackage{amssymb}
\usepackage{mwe} % for blindtext and example-image-a in example
\usepackage{wrapfig}
\graphicspath{{figs/}}
\providecommand{\mydet}[1]{\ensuremath{\begin{vmatrix}#1\end{\vmatrix}}}
\providecommand{\myvec}[1]{\ensuremath{\begin{bmatrix}#1\end{\bmatrix}}}
\providecommand{\cbrak}[1]{\ensuremath{\left\{#1\right\}}}
\providecommand{\brak}[1]{\ensuremath{\left(#1\right)}}
\providecommand{\pr}[1]{\ensuremath{\Pr\left(#1\right)}}
\providecommand{\qfunc}[1]{\ensuremath{Q\left(#1\right)}}
\providecommand{\sbrak}[1]{\ensuremath{{}\left[#1\right]}}
\providecommand{\lsbrak}[1]{\ensuremath{{}\left[#1\right]}}
\providecommand{\rsbrak}[1]{\ensuremath{{}\left.#1\right}}
\providecommand{\brak}[1]{\ensuremath{\left(#1\right)}}
\providecommand{\lbrak}[1]{\ensuremath{\left(#1\right.}}
\providecommand{\rbrak}[1]{\ensuremath{\left.#1\right)}}
\providecommand{\cbrak}[1]{\ensuremath{\left\{#1\right\}}}
\providecommand{\lcbrak}[1]{\ensuremath{\left\{#1\right.}}
\providecommand{\rcbrak}[1]{\ensuremath{\left.#1\right\}}}

\begin{document}
\begin{enumerate}


\item A solid sphere is cut into two hemispheres. The ratio of the surface areas of the sphere to that of the two hemispheres taken together is:
	\begin{enumerate}    
\item $1:1$
    \item $1:4$
    \item $2:3$
    \item $3:2$
	\end{enumerate}

\item The volume of the largest right circular cone that can be carved out from a solid cube of edge $2 \, \text{cm}$ is:
	\begin{enumerate}    
\item $\frac{4\pi}{3} \, \mathrm{cu cm}$
    \item $\frac{5\pi}{3} \, \mathrm{cu cm}$
    \item $\frac{8\pi}{3} \, \mathrm{cu cm}$
    \item $\frac{2\pi}{3} \, \mathrm{cu cm}$
	\end{enumerate}

\item \textbf{Assertion (A):} The tangents drawn at the end points of a diameter of a circle are parallel.

\textbf{Reason (R):} The diameter of a circle is the longest chord.
\begin{enumerate}
    \item Both Assertion (A) and Reason (R) are true, and Reason (R) is the correct explanation of Assertion (A).
    \item Both Assertion (A) and Reason (R) are true, but Reason (R) is not the correct explanation for Assertion (A).
    \item Assertion (A) is true, but Reason (R) is false.
    \item Assertion (A) is false, but Reason (R) is true.
\end{enumerate}

\item $AD$ is a median of $\triangle ABC$ with vertices $A(5, -6)$, $B(6, 4)$, and $C(0, 0)$. The length of $AD$ is equal to:
	\begin{enumerate}    
\item $\sqrt{68}$ units
    \item $2\sqrt{15}$ units
    \item $\sqrt{101}$ units
    \item $10$ units
	\end{enumerate}

 \item If the distance between the points $\brak{3, -5}$ and $\brak{x, -5}$ is 15 units, then the values of $x$ are:
    \begin{enumerate}
    \item $12, -18$
    \item $-12, 18$
    \item $18, 5$
    \item $-9, -12$
    \end{enumerate}

    \item The center of a circle is at $\brak{2, 0}$. If one end of a diameter is at $\brak{6, 0}$, then the other end is at:
	\begin{enumerate}    
		\item $\brak{0, 0}$
		\item $\brak{4, 0}$
		\item $\brak{-2, 0}$
		\item $\brak{-6, 0}$
	\end{enumerate}

\end{enumerate}
\end{document}
